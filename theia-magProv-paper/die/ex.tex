\section{{\LaTeX} examples}
\label{s:ex}

\paragraph{Commands.}
% use \sys
\sys helps students to use {\LaTeX} for writing a paper.
To comment on the paper, use
\TODO{to do.}
\XXX{fix this.}
\TK{it's mine.}

\paragraph{Pointing.}
% how to refer
Use \autoref{s:ex} to refer a section, 
use \autoref{t:simple} for a table,
and use \autoref{f:turtle} for a figure. 
To cite, use~\cite{kim:userfs,kim:poirot}. 
To refer a website, use \url{http://taesoo.org}. 

\paragraph{Forms.}

% compact list, [leftmargin=.2in]
\begin{itemize}[noitemsep,nolistsep]
 \setlength{\itemsep}{-0pt}
 \item foo
 \item bar
 \item baz
\end{itemize}

\paragraph{Generated figures.}

To render a svg, put the file to \cc{fig}, and include 
it to the latex like below. 

\begin{figure}[h]
\centering
\footnotesize
\includegraphics[width=0.5\columnwidth]{fig/ex-turtle}
\caption{A simple figure, generated from svg.}
\label{f:turtle}
\end{figure}

Similarly, to render a code, put a code snippet to \cc{code}, 
and include it to the latex like below.

\begin{figure}[h]
\centering
\footnotesize
\input{code/ex-code.py}
\caption{A highlighted code snippet, generated from 
 \cc{data/ex-code.py}
}
\label{f:code}
\end{figure}

Similarly, to render a \cc{gnuplot}, put a \cc{gnuplot}
script to \cc{data} with your actual data, 
and include it to the latex like below.

\begin{figure}[h]
\centering
\footnotesize
\includeplot{data/ex-gaussian}
\caption{A simple plot, generated directly from \cc{gnuplot}.}
\label{f:gaussian}
\end{figure}

\begin{figure}[h]
\centering
\footnotesize
\includeplot{data/ex-plot}
\caption{A simple plot, generated from dat.}
\label{f:plot}
\end{figure}

\paragraph{Table.}

% param: (cap-> force)
%  {h/H}ere
%  {t/T}op
%  {b/B}ottom
\begin{table}[h]
\centering
\footnotesize
\input{fig/ex-simple}
\caption{A simple table.}
\label{t:simple}
\end{table}

\begin{table*}[t]
\centering
\footnotesize
\input{fig/ex-big}
\caption{A big table.}
\label{t:big}
\end{table*}
